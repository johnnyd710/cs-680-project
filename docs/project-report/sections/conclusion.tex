\section{Conclusion}

% advantages of SOM

The raw vibration signal itself is too noisy and thresholds will result in too many alarms.
To alleviate this, using features such as the moving average or the standard deviation of the signal can help.
Still, these methods only analyze one signal at a time and lack robustness.
A SOM factors multiple variables into one metric and eliminates the need for manual signal analysis or thresholds.
It also beats traditional methods in terms of number of alarms and percentage of time above threshold.
There was no opportunity to compare methods in terms of early prediction because all of the data sets provided to us were binary (a data set was either `healthy' or `unhealthy').
Despite this, there is evidence suggesting SOMs can in fact give early prediction of bearing failures \cite{Tian2014AnomalyDU} \cite{som-1}.
This should be the focus on future work, because the ultimate goal is to predict failure, not just detect it while it is happening.
One way to achieve this would be to forecast the health score.
Another way could be to try supervised methods such as logistic regression.
Still, this application proved SOMs can be used on real data from industry for condition maintenance.
Dealing with real industry data can be challenging due to noise and countless other factors.
For example, operators at the steel mill may have lubricated the bearing resulting in temporary health improvement.

Overall, the use of this application at a steel manufacturer would give operators a sense of how necessary maintenance is at a particular time,
encouraging only performing maintenance when the bearing is unhealthy.
This would avoid unintended consequences such as the Waddington Effect.
Even better, this could give operators the opportunity to change the bearing during a planned shutdown, avoiding a costly unplanned shutdown later on. 
This could save the plant millions of dollars in lost production.