% who what where when
\section*{Motivation}

An often overlooked part of an asset's expenses is maintenance.
A popular maintenance strategy is known as Preventative Maintenance (PvM), 
where maintenance is regularly performed on an asset while it is still in good condition to prevent it from breaking down unexpectedely.
Over \$ 200 billion is spent on such maintenance every year in the United States and one-third is wasted on improper or unnecessary maintenance \cite{mobley2002introduction}.
Even worse, no maintenance at all can lead to unexpected failures which in turn can cause serious economic consquences or injury.
There exists a significant need to modernize maintenance techniques around the world to ensure safety, reliability, and efficiency.

During the Second World War, a British scientist named Conrad Waddington made a fascinating discovery about the maintenance of aircraft while working for the Royal Air Force (RAF).
Previously, aircraft bombers had a notorious problem of breaking down - in fact the ideal serviceabitity in a squadron of bombers was only around 70-75\% \cite{Morse1364}.
What he discovered was that preventative maintenance methods actually increased the rate of unexpected failure.
The process of more maintenance leading to more failures became known as the Waddington Effect as a result.
By increasing the interval between maintenance cycles and eliminating all maintenance deemed unnecessary,
Waddington was able to increase the effective flight hours of the RAF bomber fleet by 60\% \cite{Morse1364}.

After this discovery, asset owners around the world tried to find the optimal time to repair an asset.
This led to the invention of Predictive Maintenance (PdM), a philosphy that uses the actual operating condition of assets to optimize operations \cite{mobley2002introduction}.
For PdM to be effective, the asset's operating condition must be estimated.
Estimating the asset's operating condition is the focus of the paper, which is refered to as the \textit{health score}.