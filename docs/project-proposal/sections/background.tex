\section{Background} % Numbered section

%------------------------------------------------
\subsection{Why Machine Learning?}
% why ML?
One of the most common reasons PdM methods fail is a lack of continuous improvement and a lack of repeatability \cite{whypdmfails}.
Additionally, equipment monitoring is a time consuming process, requires experts to identify failure patterns, and is expensive.
Machine learning provides an automated approach that requires minimal asset knowledge, is inexpensive, can be trained on many assets, and can be re-trained as operating conditions change.

%Machine learning techniques for predictive maintenance were considered not practical, too complex, or too time consuming.
%In particular, plant managers did not want to change their existing infrastructure (the software that handles data acquisition and analyzes it) to adopt the technology.
%But now Asset Performance Management (APM) software providers are growing and condition based maintenance is at the forefront.
%As they team up with cloud based data solutions, it becomes easy for asset operators to implement machine learning in their existing data infrastructures via a simple call to the cloud.
%Manufacturers around the world use APM technology from Bentley Systems, a global leader in APM capabilities according to a recent Gartner report \cite{foust_steenstrup_2018}.
%The proposed solution will be deployed and maintained using APM software from Bentley Systems.

%\subsection{ArcelorMittal Dofasco}
\subsection{Steel Manufacturing}
% Who is dofasco
%ArcelorMittal Dofasco is a steel company located in Hamilton, Ontario.
%They use Bentley's APM software and are eager for a machine learning solution to detect the operating conditions of various assets. 
%In particular, they have offered a real data set of several industrial level furnace fans located in the Hamilton plant. 
%Specifically, the data is composed of several smaller data sets, each representing various hours of operation.
%See the Appendix for full list of variables included.

% Steel mill 
In a steel making plant, called a steel mill, operations run almost 24/7 except when the mill is shut down once a month for repairs and maintenance.
A failure of an asset leading to a shutdown at any other time results in severe costs.
If the operating condition of the asset is known, then operators can determine whether or not it should not be repaired during that scheduled downtime,
thus avoiding costly unexpected failures and the Waddington Effect.


%What is a reheating furnace?
One of the most costly failues occurs when the fans for the reheating furnance fail.
A reheating furnace is used to raise the internal temperature of steel, so that it can be shaped into a final product.
Setting the correct temperature is one of the most essential factors of product quality in the plant.
The temperature is so high that if at any point the blast furnaces fail, the entire line must be shut down for days to allow the furnance to cool before operators inspect the cause of failure.
%A subject matter expert from ArcelorMittal suggests this could cost millions of dollars in lost production.
A subject matter expert from the steel manufacturer suggests this could cost millions of dollars in lost production.

% about the failure
%One way these furnaces fail is via a failure of the exhaust fans.

%It is therefore critical to steel manufacturers to replace the exhaust fans before a failure occurs.

%------------------------------------------------
