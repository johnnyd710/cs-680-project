%%%%%%%%%%%%%%%%%%%%%%%%%%%%%%%%%%%%%%%%%
% Lachaise Assignment
% LaTeX Template
% Version 1.0 (26/6/2018)
%
% This template originates from:
% http://www.LaTeXTemplates.com
%
% Authors:
% Marion Lachaise & François Févotte
% Vel (vel@LaTeXTemplates.com)
%
% License:
% CC BY-NC-SA 3.0 (http://creativecommons.org/licenses/by-nc-sa/3.0/)
% 
%%%%%%%%%%%%%%%%%%%%%%%%%%%%%%%%%%%%%%%%%

%----------------------------------------------------------------------------------------
%	PACKAGES AND OTHER DOCUMENT CONFIGURATIONS
%----------------------------------------------------------------------------------------

\documentclass{article}

\input{structure.tex} % Include the file specifying the document structure and custom commands

%----------------------------------------------------------------------------------------
%	ASSIGNMENT INFORMATION
%----------------------------------------------------------------------------------------

\title{CS 680: Project Proposal} % Title of the assignment

\author{John DiMatteo\\ \texttt{jdimatteo@uwaterloo.ca}} % Author name and email address

\date{University of Waterloo--- \today} % University, school and/or department name(s) and a date

%----------------------------------------------------------------------------------------

\begin{document}

\maketitle % Print the title

%----------------------------------------------------------------------------------------
%	INTRODUCTION
%----------------------------------------------------------------------------------------

\begin{center}
\section*{Abstract}
It remains critical for several important industries to have knowledge of asset health.
A \textit{Condition Monitoring} strategy using machine learning is proposed as a research project for CS 680.
An algorithm will be developed and evaluated on several large scale reheat furnace fans provided by a steel manufacturer.
\end{center}

\newpage

\section*{Introduction} % Unnumbered section

The following is a proposal for a graduate research project for the University of Waterloo Computer Science class titled ``CS 680: Introduction to Machine Learning''.
The goal is to use machine learning to give early warning to operators of an asset in case of failure.
This will allow operators to repair or replace the asset with a scheduled maintenance task, as opposed to unexpected failure resulting in severe downtime costs.
A proposed solution, using unsupervised machine learning, is to provide users with an indication of the health of a particular asset (known as \textit{condition monitoring}).
The machine learning algorithm will be implemented and evaluated on an industrial furnace fan data set provided by a steel manufacturer.

\subsection*{Problem Statement}
\begin{center}
	\textit{
To provide users with an indication of asset failure, allowing for cost savings by early maintenance or replacement.
	}
\end{center}
%----------------------------------------------------------------------------------------
%	BACKGROUND
%----------------------------------------------------------------------------------------

\section{Background} % Numbered section

%------------------------------------------------

\subsection{Blast Furnances}


%What is a blast furnace?
Steel manufacturers typically have X blast furnaces in a mill.
They are used to raise the internal temperature of steel.
Setting the correct temperature is critical to product shape and quality.
If at any point the blast furnaces fail, the entire line must be shut down to allow operators to find and repair the furnaces.
Operators suggest this could take X amount of time.
During this time, all steel production is stopped.
That is why it is crucial for operators to have early warning so repairs can be made during scheduled shutdowns and not during live operation.

% fans
One way these furnaces fail is via a failure of the exhaust fans.


% Predictive maintenance ...
A predictive maintenance solution should 

%------------------------------------------------

\subsection{Previous Work}

Using machine learning for condition monitoring is not new.

% supervised learning approach and problems

% one class SVM and problem (no score)

% autoencoders 

% self organized maps


% Math equation/formula
\begin{equation}
	I = \int_{a}^{b} f(x) \; \text{d}x.
\end{equation}


%----------------------------------------------------------------------------------------
%	PROBLEM 2
%----------------------------------------------------------------------------------------

\section{Implementation}

% unsupervised anomaly score

% KPI


%----------------------------------------------------------------------------------------
\section{Evaluation}


\bibliography{bibliography}
\bibliographystyle{plain}

\end{document}
